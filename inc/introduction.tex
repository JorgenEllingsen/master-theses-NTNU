\chapter{Introduction}
\label{chap:introduction}
In an increasingly digital world, information is valuable for governments and industry alike. Digital industry leaders has built their business on targeted marketing and big data for years, and the public is slowly realizing the scope of their digital identity and the consequences of not owing their own information. The General Data Protection Regulation (GDPR) is taking a large step to ensure that governments and industry is managing personal information correctly, and that the individual is in control of their own information. 

As more and more businesses and governmental entities understand the value of information, we're faced with a new challenge in information security and individual freedom. We're often required to provide proof of identity on physical locations when buying beer or renting a car. This work will explore the feasibility of using distributed ledgers to manage identities by segregating attributes and transfer control of correlation back to the individual. When entering a bar to meet an old friend, the establishment is required to ensure that you're old enough to buy alcohol, but should not automatically be entitled to information like your name and social security number - other attributes often visible on government issued ID cards. 

Several papers on identity management in distributed ledgers has been published in the past, and this research will focus on implementation and management of attribute based identity verification in a physical interaction. 

The challenge around separating attributes are partially solved by previous papers, but for this to be a working solution as a cyber physical identity service, the system must be cheap and fast in order to work in day-to-day activities. In addition, this thesis will suggest what is required both from the device of the Service Provider and from device of the User, and a analysis of the feasibility of using a mobile or a wearable device for this interaction.

Governments and private companies are currently in a position were they control too much information about our private and social life, movement and habits. When you're asked for ID when entering a bar today, you're effectively disclosing social security number, full name, full date of birth, and other attributes about your identity - when the establishment actually only require to know if you are over 18 years old. The same is valid for governmental controls, like driving license - a police officer has pulled you over, without suspicion of any illegal activity, must control that you do indeed have a drivers license, but is not required to know who you are, and what you're doing. 

A distributed ledger is in its nature visible to anyone who care to look, and the attributes of an identity must therefore be cryptographically separated such that only the individual themselves can connect the information and prove that the information is verified by a trusted identity provider. 

The technology of distributed ledgers, and the research done on identity management in blockchain, enables further research were identity providers and identity holders can agree upon attributes that are available only if the identity holder wants to share them - in a system capable of providing secure proof of identity by verified attributes. This can bring the control of information back to the individual, providing privacy and the possibility of revocation of personal information while the identity attributes are still verifiable as genuine from the given identity provider.

The research problem is in this paper to see if a distributed ledger identity management system can be altered or improved such that it is suitable as physical proof of an identity attribute. This will include finding a way to make the current identity management proposals cheap and fast enough to be a viable option for identity management, and analyze what type of devices is needed to be carried around by the individual to physically prove that they have the required attribute(s). Furthermore, both the user and the service provider must be able to trust the system, and would most likely carry a device each, there is a need to determine what controls must be done on each party's device, and how this information is communicated between them.

A solution were an individual could prove one or more attributes to a entity without disclosing their full identity would be beneficial to privacy and anonymity both online and offline. Utilizing the technology for distributed ledgers like Blockchain~\cite{bitcoin2008} or Tangle~\cite{IOTA_Whitepaper}, we can establish such a mechanism by building on the clever use of cryptographic algorithms proposed in previous papers. The blockchain and the Tangle will be discussed later in Section~\ref{related:work}. This will strengthen the control each individual have over their identity and shared information, and will still provide the service providers with the required information. \cite{Azouvi2017} identifies that their solution does not prove zero-knowledge proofs, but relies on lightweight cryptographic primitives. An identity provider will always know the information contained in the attribute, but should not be able to know if and when the attribute was accessed, and for what purpose. Given the example of entrance to a bar, there should be possible to prove that the individual is over a certain age, but the establishment would not require the exact birth date, and the identity provider would not need to know that the individual shared that information.

While Identity Management in distributed ledgers have been explored in the past, it is usually on either the Bitcoin or Etherium blockchain~\cite{Azouvi2017,Augot2017}.  In addition, the previous research have focused on establishing and maintaining an online identity, but does not explore the possibility of physically presenting proof of an online identity.

This thesis will focus on how an identity can be verified in a physical location, and with that there are some new challenges to overcome. This will require almost instant verification, something that likely will require a different ledger than those previously explored for Identity Management. It will also require the User to carry some sort of identity claim, and the Service Provider to have some way of verifying this claim. 

When these challenges are solved, a prototype will be made as a proof of concept. While this prototype can not be planed in detail before a choice of distributed ledger is made, it is likely to be built as a web application such that it can be easily ported to any laptop or smartphone for testing.