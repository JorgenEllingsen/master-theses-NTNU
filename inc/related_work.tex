\chapter{Related Work}
\label{chap:related_work}
Several academic papers propose Identity Management in blockchain applications, and the proposed implementation described in~\cite{Augot2017} provides functionality close to what is required in this system. They propose three types of actors; \textit{Identity Providers} (IP), \textit{Service Providers} (SP) and \textit{Users} (USR), and require three steps in their protocol; \textit{Setup Phase}, \textit{Enrollment Phase}, and \textit{Operational Phase}. In the \textit{Setup Phase}, the \textit{Identity Provider} chooses some set of attributes and makes them publicly available on the Bitcoin ledger. In the \textit{Enrollment Phase}, a \textit{User} brings proof of identity to the \textit{Identity Provider} (Physically or virtually, based on the policy of the \textit{Identity Provider}) that verifies all the attribute values of the claimed identity. This is finalized with a single transaction to the bitcoin ledger, that is considered a \textit{Authentication token}.  In the \textit{Operational Phase} the \textit{User} issues a transaction with the authentication token with outputs to both the \textit{Identity Provider} and back to itself for future transactions. The \textit{Service Provider} issues an Acknowledgement of the identity by sending output from the transaction to the \textit{Identity Provider}. The system is more complicated than described here, with possibilities for revocation and suggestions for storage outside of the blockchain by including the hash of the information in the \textit{OP\_RETURN} instead of the data itself. 

In \cite{Augot2017} the system relies on a Discrete Logarithm Representation (DLREP) proposed in \cite{Brands2000} to efficiently reveal selected parts of an identity to verifiers, while any other information remains hidden. The DLREP can be used to prove boolean functions about the identity, and will satisfy this systems requirement of privacy. 

In~\cite{Azouvi2017} the authors propose several solutions to register identities and attributes in a system built on public ledgers and compare them in terms of privacy, usability and integrity. Two of their solutions satisfy attribute integrity and privacy, and are named \textit{Multi-Casascius} and \textit{Mix-Network}. Both these solutions provide for passive verification, something the authors describes as the possibility to verify the identity only by looking at the public ledger - without the need for additional transactions or information from external sources.
